% Chapter 1
\part{Reductions} % Main chapter title

\label{chap:3}


\chapter{Development}
\section{Development Environment}

In this chapter we briefly introduce the development environment used to program reductions.\\

We use the programming language \texttt{Python} for three main reasons:

\begin{itemize}
\item \texttt{Python} is interpreted. Furthermore, it is compatible with major platforms and systems. This allow us to develop a program that is accessible to virtually everyone without the need of compiling the program multi-platform. 

\item \texttt{Python} is considered to have one of the most accessible learning curves. Therefore, is excellent to introduce to reductions programming to those who are mostly interested in the theoretical part.

\item \texttt{Python}  and all necessary software dependencies is  free software\cite{stallman2002free}. I believe that, whenever possible, public-funded research should be accessible to everyone interested. This implies reducing the barriers to access to knowledge to a minimum.  Furthermore, it must not encourage the economic gain of a third party private entity that is not even aware of the existence of the project by making it compulsory to use it for the full exploitation of the project carried out.
\end{itemize}



All test cases are done in an ARCH-BASED OS. In particular:

  {\begin{center}
      \texttt{Linux 4.19.122-1-MANJARO  2020 x86\_64\ GNU/Linux}\\
    \end{center} }

  With 4 physical (8 virtual) processor \texttt{Intel(R) Core(TM) i7-8550U CPU @ 1.80GHz}. 

  
\section{PySAT}
PySAT\cite{imms-sat18} is a library developed for python SAT solving. It has been primarily develop by Alexey Ignatiev, Antonio Morgado, Joao Marques-Silva since 2018. Among their feature we can highlight:

\begin{itemize}
\item Solvers:  PySAT include some solvers of great diffusion. Namely
  \begin{itemize}
  \item CaDiCaL: A CDCL based, developed by Armin Biere et al. Armin Biere is one of the leading voices in SAT Solving. Is one of editors of Handbook of Satisfiability. CaDiCaL won the SAT Race 2019.
  \item Maplesat: 
  \item Minisat
  \end{itemize}

  
\item Cardinality Encodings:

\end{itemize}


\section{Path based Problems}


 In order to demonstrate the utility a series of reductions will be developed. This will imply a formal approach to the resolution of the problems, as well as deploying a little theoretical background to some problems when needed. Also we would like to show that this technique provide sometimes really simple approximations to the problems.\\

\section{Hamiltonian Cycle}
\subsection{Resolution}

The problem of, given a graph, find a Hamiltonian Cycle is well know to be NP-Complete. Then by Cook Theorem it is known that a reduction from the problem of the Hamiltonian Cycle to SAT exists. This theorem is constructive, so it effectively does give a reduction. Nonetheless, this reduction is unmanageable and in order to use SAT-solvers to improve Hamiltonian cycle resolution it would be necessary to improve it. On this subsection an alternative reduction will be proven.

\begin{definition}
  A Hamiltonian cycle is a cycle that visit every node in a graph. The associated problem is to check, given a graph, whether whether cycle exists.
\end{definition}

We will consider the problem of the Hamiltonian cycle of undirected graphs. Therefore an edge would have two sources instead of a source and a target as it is regarded on directed graphs.\\

This problem is a very good example to represent what mean to use a SAT-solver to solve a hard problem. The presented reduction is done as shown in \cite{49593}, with a minor error solved. It move the complexity of the problem from how to solve it to how to implement a SAT-solver. Therefore it only left a worry about what do I need to satisfy in order to solve this problem In order to make the reduction we will represent with Boolean clauses the conditions:\\

Let $G=(V=\{ v_1,...,v_n\},E= \{e_1,...,e_m\})$ be a graph. To reduce it to a SAT problem, we will first define the variables $\{x_{i,j}: i\in 1,...,n ; j\in 1,...,n+1 \}$. If the variable $x_{i,j}$ is assign to true it would mean that the vertex $v_i$ is in position $j$ in the path. We would like to find a assignment of these variables that satisfy the following clauses:

\begin{enumerate}
\item Each vertex must appear at least once in the path, for every vertex $v_i$:
  $ x_{i,1} \vee ... \vee x_{i,n +1},\ i \in 1,...,n$.
\item Each vertex must not appear twice in the path, unless it is the first and last node: $\neg x_{i,j} \vee \neg x_{i,k},\ i \in 1,...,n ,\ j \in 2,...,m+1, k \in 1,...,m$.
\item Every position in the path must be occupied:   $ x_{1,i} \vee ... \vee \x_{n,i},\ i \in 1,...,m+1$.
\item Two consecutive vertex have to be adjacent: $\neg x_{i,j} \vee \neg x_{i+1,k}\ \forall (k,j) \not \in E$.
\end{enumerate}

Let now solve that this is a correct reduction, i.e., that an assignment that can satisfy these clauses exists if, and only if, the graph $G$ has a Hamiltonian graph. If such an assignment exists we can make a Hamiltonian cycle with the variables assigned to 1. On the other hand if such cycle exists an assignment that assign to 1 the variable $x_{i,j}$ given that the vertex $v_i$ is in position $j$ in the path would satisfy all the clauses. 

\subsection{Implementation}

In this subsection, once we have proved the reduction in a constructive way, we proceed to program this in order to make this ideas usable.

\subsubsection{Planning and Budgeting}



\subsubsection{Analysis and Design}

\subsubsection{Test}

