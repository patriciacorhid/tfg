% Chapter 1
\part{Reductions} % Main chapter title

\label{chap:3}



\chapter{Development Enviroment}

In this chapter we briefly introduce the development 

\section{PySAT}

\section{A competition winning solver: MAPLE-CM}



\chapter{Path based Problems}


 In order to demonstrate the utility a series of reductions will be developed. This will imply a formal approach to the resolution of the problems, as well as deploying a little theoretical background to some problems when needed. Also we would like to show that this techniches provide sometimes really simple approximations to the problems.\\

\section{Hamiltonian Cycle}
\subsection{Resolution}

The problem of, given a graph, find a Hamiltonian Cycle is well know to be NP-Complete. Then by Cook Theorem it is known that a reduction from the problem of the Hamiltonian Cycle to SAT exists. This theorem is constructive, so it effectively does give a reduction. Nonetheless, this reduction is unmanageable and in order to use SAT-solvers to improve Hamiltonian cycle resolution it would be necessary to improve it. On this subsection an alternative reduction will be proven.

\begin{definition}
  A Hamiltonian cycle is a cycle that visit every node in a graph. The associated problem is to check, given a graph, whether whether cycle exists.
\end{definition}

We will consider the problem of the Hamiltonian cycle of undirected graphs. Therefore an edge would have two sources instead of a source and a target as it is regarded on directed graphs.\\

This problem is a very good example to represent what mean to use a SAT-solver to solve a hard problem. The presented reduction is done as shown in \cite{49593}, with a minor error solved. It move the complexity of the problem from how to solve it to how to implement a SAT-solver. Therefore it only left a worry about what do I need to satisfy in order to solve this problem In order to make the reduction we will represent with Boolean clauses the conditions:\\

Let $G=(V=\{ v_1,...,v_n\},E= \{e_1,...,e_m\})$ be a graph. To reduce it to a SAT problem, we will first define the variables $\{x_{i,j}: i\in 1,...,n ; j\in 1,...,n+1 \}$. If the variable $x_{i,j}$ is assign to true it would mean that the vertex $v_i$ is in position $j$ in the path. We would like to find a assignment of these variables that satisfy the following clauses:

\begin{enumerate}
\item Each vertex must appear at least once in the path, for every vertex $v_i$:
  $ x_{i,1} \vee ... \vee x_{i,n +1},\ i \in 1,...,n$.
\item Each vertex must not appear twice in the path, unless it is the first and last node: $\neg x_{i,j} \vee \neg x_{i,k},\ i \in 1,...,n ,\ j \in 2,...,m+1, k \in 1,...,m$.
\item Every position in the path must be occupied:   $ x_{1,i} \vee ... \vee \x_{n,i},\ i \in 1,...,m+1$.
\item Two consecutive vertex have to be adjacent: $\neg x_{i,j} \vee \neg x_{i+1,k}\ \forall (k,j) \not \in E$.
\end{enumerate}

Let now solve that this is a correct reduction, i.e., that an assignment that can satisfy these clauses exists if, and only if, the graph $G$ has a Hamiltonian graph. If such an assignment exists we can make a Hamiltonian cycle with the variables sasigned to 1. On the other hand if such cycle exists an assignment that assign to 1 the variable $x_{i,j}$ given that the vertex $v_i$ is in position $j$ in the path would satisfy all the clauses. 

\subsection{Implementation}

In this subsection, once we have proved the reduction in a constructive way, we proceed to program this in order to make this ideas usable.

\subsubsection{Planning and Budgeting}



\subsubsection{Analysis and Design}

\subsubsection{Test}

