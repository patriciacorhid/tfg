
\chapter{Resolution Algorithms}

\section{Special Cases Solvable in PolynomialTime}

In this section we will discuss some cases of the sat problem solvable in P. These cases are of interest because polynomial is no achievable in all cases.

\begin{definition}
  Let $F$ be a formula. A subset $ V \subset Var(F)$ is called a backdoor if $F\alpha \in \text{P}$ for every assigment $\alpha$ that maps all $V$.
\end{definition}
A goal for a SAT-solver could be to find a backdoor of minimun size. DPLL would try to search for a backdoor, using heuristics in order not to explore all subsets (only achievable if such backdoor exists).
\subsection{Unit Propagation}

Unit propagation is a simple concept that is worth standing out because it would be comonplace. Given a CNF formula $F$ if there is a clause with only one element that should be assigned acordingly to the clause, otherwise $F$ is unsatisfiable. This lead to the unit propagation concept. Whenever we have a unitary clause $\{p\}$ we should \emph{resolve} it and start working with $F[p=1]$ being $[p=1]$ the assigment that maps the value of the metavariable $p$ to 1, which could posibly imply mapping a variable to $0$.

Also, the unit propagation might result on a recursive problem, as other unit clauses could appear.

\subsection{2CNF}
It is already know that 3CNF is equivalent to SAT. This is not known for 2CNF and is belived to be false.

\begin{proposition}
  2CNF is in P 
\end{proposition}
\begin{proof}

  To prove that 2CNF is in P, an algorithm polinomial on the number of clauses will be given. Let $F \in$ 2CNF.  Without loosing of generality, we will consider that there are no clauses in $F$ $\{u,u\}$ or $\{u,\neg u\}$ as the first one should be handle with unit propagation and the second one is a tautology. Therefore each clause is $(u \vee v)$ with $var(u) \ne var(v)$, which could be seen as $(\neg u \rightarrow v) \wede (\new v \rightwarrow u)$.\\


  
  We would consider a step to be as follow: we choose a variable $x \in Var(F)$ and set it to 0. Them a chain of implication would arise, which might end on conflict. If no conflitc arise, then is an autark assigment, so repeat the process. Otherwise set it to 1 and proceed. If conflitc arise, then $F$ is unsatisfiable. If no conflitc arise, then is an autark assigment, so repeat the process.
  

  Each step is of polynomial time over the number of clauses. Also there would be at most as many steps as variables, therefore we have a polynomial algorithm.
  
  
\end{proof}

\subsection{Horn Formulas}

In this subsection we will analize Horn formulas. They named after Alfred Horn, who defined them on his work\cite{horn1951sentences}. They are of special intereset as is HORNSAT is P-complete.


\begin{definition}
  Let $F$ be a formula in CNF. It is said to be a horn formula if for every $C \in F$ there is at most one non-negated literal. HORN will be the set of all horn formulas.

  HORNSAT will be the intersection of HORN and SAT problems. Nonetheless, given the easiness of checking whether a formula is in HORN, it would usually consider as the problem that check the satisfability of a horn formula.
\end{definition}


\begin{proposition}
  HORNSAT is in P.
\end{proposition}
\begin{proof}
  Given a formula it could have a clause with only one non-negated literal or not have it. If it does not have a clause like this, set all the variables to 0 and is solved. Otherwise, unit-propagate the unary clause and repeat the process, as it would necesary be done to solve the problem. If a contradiction is raised, them the formula is not satisfiable.
\end{proof}


Now we will discuss a powerfull generalization of Horn formulas: the renamable Horn Formulas. 

\begin{definition}
  Let $F$ be a CNF formula. $F$ is called renamable Horn if there is a subset $U$ of the variables $Var(F)$, so that $F[x=\neg x | x \in U]$ is a Horn formula.
  That set would be called a renaming.
\end{definition}

In order to prove an important result on renaming, we will introduce some notation.

\begin{definition}
  Let $F$ be a CNF formula. Then a 2CNF formula $F^*$ is defined as:
  $$F^* = \{(u \vee v) | u,v \text{are literals in the same clause} K \in F \}$$
\end{definition}


\begin{theorem}
  The CNF formula $F$ is renamable Horn if and only if the associated $F*$ formula is satisfiable. Moreover, if satisfying assigment $\alpha$ for $F^*$  exists then it encodes a renaming $U$ in the sense that $x \in U \iff \alpha(x) = 1$.
\end{theorem}
\begin{proof}
  Let $F$ be renamable Horn a $U$ be a renaming. We consider the assigment $\alpha$ that map to 1 all variable in $U$ and map to 0 otherwise. Let $\{u\vee v\} \in F^*$ after the renaming. There should be at least one negative variable so if every variable is set to 0, $F^*$ is satisfiable.

  The other direction is analogous: let $\alpha$ be an assigment that satisfie $F^*$. Them there is no to literals in the same clause set to 0. Defining $U=  {x \in Var(F) | \alpha(x) = 1}$ there is no two positives variables in a clause.
  \end{proof}


These mean that if a renaming exists, if could be obtained efficiently.






