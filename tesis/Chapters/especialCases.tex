\newpage

\section{Special Cases Solvable in PolynomialTime}


In this section we will discuss some cases of the sat problem solvable in P. These cases are of interest because polynomial is no achievable in all cases.

\begin{definition}
  Let $F$ be a formula. A subset $ V \subset Var(F)$ is called a backdoor if $F\alpha \in \text{P}$ for every assigment $\alpha$ that maps all $V$.
\end{definition}
A goal for a SAT-solver could be to find a backdoor of minimun size. DPLL would try to search for a backdoor, using heuristics in order not to explore all subsets (only achievable if such backdoor exists).
\subsection{Unit Propagation}


\subsection{2CNF}
It is already know that 3CNF is equivalent to SAT. This is not known for 2CNF and is belived to be false.

\begin{proposition}
  2CNF is in P 
\end{proposition}
\begin{proof}

  To prove that 2CNF is in P, an algorithm polinomial on the number of clauses will be given. Let $F \in$ 2CNF.  Without loosing of generality, we will consider that there are no clauses in $F$ $\{u,u\}$ or $\{u,\neg u\}$ as the first one should be handle with unit propagation and the second one is a tautology. Therefore each clause is $(u \vee v)$ with $var(u) \ne var(v)$, which could be seen as $(\neg u \rightarrow v) \wede (\new v \rightwarrow u)$.\\


  
  We would consider a step to be as follow: we choose a variable $x \in Var(F)$ and set it to 0. Them a chain of implication would arise, which might end on conflict. If no conflitc arise, then is an autark assigment, so repeat the process. Otherwise set it to 1 and proceed. If conflitc arise, then $F$ is unsatisfiable. If no conflitc arise, then is an autark assigment, so repeat the process.
  

  Each step is of polynomial time over the number of clauses. Also there would be at most as many steps as variables, therefore we have a polynomial algorithm.
  
  
\end{proof}

\subsection{Horn Formulas}
\subsection{Renamable horn}


