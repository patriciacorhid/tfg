

\chapter*{}
\section*{Conclusions} 
\addcontentsline{toc}{chapter}{Conclusions and further work}


In this document we have developed an overview on the properties, algorithms and applications of SAT problems. We have made a selection of these results, mainly because the investigation on SAT has been a thriving field in the last half century. Nonetheless, we consider that all main results have been covered. The algorithms studied could also be celebrated, as we consider to have studied the most influential algorithms presents in literature.\\

We are satisfied with the work done in the bibliographic research on the topic. On the development of the library we celebrate what is done, with the note that much more can be done. There are a lot of NP-Complete problems, and as many reductions.


\section*{Further work}

We propose further work to be done in each part:
\begin{enumerate}
\item For the first part we are interested in the Valiant–Vazirani theorem. This theorem states that if there is a polynomial algorithm for Unambiguous-SAT, then NP = RP. The proof is based on the Mulmuley–Vazirani–Vazirani isolation lemma.
\item As we mention in the summary section, we would like to explore the Stalmark's Algorithm. For the first section of this part the addition of the Schaefer's dichotomy theorem would be nice. 
\item Initially the reduction of MAXSAT to a class based of on SAT-Solver was proposed as one of the objective of this thesis. This work has already been done for PySAT as can be seen in:
  \begin{center}
    \url{https://github.com/pysathq/pysat/blob/master/examples/rc2.py}.
  \end{center}

  Therefore we propose as further work to improve the skeleton provided with the class \texttt{nativeQBF}.   
\end{enumerate}