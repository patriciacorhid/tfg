% Chapter 1

\chapter{Theoretical Introduction} % Main chapter title

\label{Chapter1} % For referencing the chapter elsewhere, use \ref{Chapter1} 

%----------------------------------------------------------------------------------------

% Define some commands to keep the formatting separated from the content 
\newcommand{\keyword}[1]{\textbf{#1}}
\newcommand{\tabhead}[1]{\textbf{#1}}
\newcommand{\code}[1]{\texttt{#1}}
\newcommand{\file}[1]{\texttt{\bfseries#1}}
\newcommand{\option}[1]{\texttt{\itshape#1}}

%----------------------------------------------------------------------------------------
\section{Definitions and first concepts.}
In this section Boolean formulas will be introduced. We first start with the basic building blocks, which collectively form what is called  the alphabet. Namely,
\begin{itemize}
\item Symbols $x,y,z$ for Boolean variables.
\item Values 0 and 1, referring to false  and true respectively. The set $\{0,1\}$ will be named as $\mathbb{B}$.

\item Boolean Operators: 
\begin{itemize}
	\item unary: $\neg $
	\item binary: $\wedge, \vee, \rightarrow, \oplus, \leftrightarrow $
\end{itemize}


We will consider $\wedge$ of greater priority than $\vee$. These operator are defined by theirs truth table:

\end{itemize}

\vspace{0.5cm}
\begin{tabular}{cccccccc}
& &
	 \begin{tabular}{c|cc}
	$\neg $ & 0 & 1 \\\hline
	             & 1 & 0
      \end{tabular} 


&
 	 \begin{tabular}{c|cc}
	$\vee$ & 0 & 1 \\\hline
	       0   & 0 & 1 \\
	       1   & 1 & 1
      \end{tabular} 
       & 
 
  	 \begin{tabular}{c|cc}
	$\wedge$ & 0 & 1 \\\hline
	       0   & 0 & 0 \\
	       1   & 0 & 1
      \end{tabular} &
	  	 \begin{tabular}{c|cc}
	$\rightarrow$ & 0 & 1 \\\hline
	       0   & 1 & 1 \\
	       1   & 0 & 1
      \end{tabular}  & 
      
 	 \begin{tabular}{c|cc}
	$\oplus$ & 0 & 1 \\\hline
	       0   & 0 & 1 \\
	       1   & 1 & 0
      \end{tabular} 
       &
      
       
 	 \begin{tabular}{c|cc}
	$\leftrightarrow$ & 0 & 1 \\\hline
	       0   & 1 & 0 \\
	       1   & 0 & 1
      \end{tabular} \\ 
\end{tabular}

\vspace{1cm}



\begin{definition}
  A Boolean formula is defined inductively:
  \begin{itemize}
  \item The constants 0 and 1 are formulas.
  \item Every variable is a formula.
  \item If $F$ is a formula, then $\neg  F$ is a formula.
  \item The concatenation with a symbol of two formulas is a formula to.
  \end{itemize}
\end{definition}

Examples of formulas are $x\vee y$ or $x_1\wedge x_2 \vee  ( x_4 \vee \neg  x_3 \wedge (x_5\rightarrow x_6) \vee 0 )$.\\


\begin{definition} Given a set $A$ it has an associated homonym problem that consists on, given an arbitrary element $e$ check if $e\in A$.	
\end{definition}


\begin{definition}
	An assignment is a function from the set of Boolean formulas to the set of Boolean formulas, on which some variables $\{x_1,...,x_n \}$ are replaced by predefined constants $\{a_1,...,a_n\}$ respectively. If none of the variables altered by an assignment $\alpha$ are present on the formula $F$ then $\alpha(F) = F$. We denote as $Var(\alpha)$ the set of those variables that receive a value from $\alpha$. Analogously, $Var(F)$ will denote the variables present on a formula $F$.
	\end{definition}

	
	One can then \emph{apply} an assignment $\alpha$ to a formula $F$, denoting it by $F\alpha=\alpha(F)$. To describe an assignment we will use a set that pairs each variable to it value, i.e. $\alpha=\{x_1\to 1,...,x_n\to 0\}$. For example given an assignment $\alpha_0 = \{x_1 \to 1, x_2\to 1, x_3 \to 0\}$ and $F_0=x_1\to (x_2\wedge x_4)$ then  $F_0\alpha_0=1 \to (1\wedge x_4)= x_4$. \\
	
	\begin{definition}
	An assignment is said to \emph{satisfies}  a formula $F$ if $F\alpha=1$ and in the case $F  \alpha = 0 $ it is said to \emph{falsifies} the statement.
\end{definition}
	
	\begin{definition}
	A formula $F$ is called \emph{satisfiable} if $\exists \alpha : F\alpha = 1.$ Otherwise it is called \emph{unsatisfiable}. The set of all satisfiable formulas is denoted as $SAT$.  The problem $SAT$ is the associated problem. 	An assignment $\alpha$ that satisfies $F$ is called a model an is denoted as $\alpha \models F$.\\
	
	
	A formula $F$ such that for every  $\alpha$ assignment happens that $F\alpha=1$ is a tautology. Given two formulas $G,F$ it is said that $G$ follows from $F$ if $G\rightarrow F$ is a tautology. \\
\end{definition}
	
	
	
\begin{definition}
	A formula $F$ is said to be in conjunctive normal form if is written as:
	$$F = C_1\wedge ... \wedge C_n$$
	Where $C_i = (u_{1,i} \vee ... \vee u_{m_i,i})$  and $u_{i,j}$ are literals, that is, variables or negated variables. The set of all formulas in conjunctive normal form is called \emph{CNF}.
\end{definition}
	
A formula in \emph{CNF} could be seen as a collection of clauses. The associated problem with \emph{CNF} is straightforward on $O(n)$. The problem that we will investigate is whether a arbitrary formula $F$ have a \emph{SAT-equivalent} \emph{CNF} formula.
Equivalently a clause could be seen as a set of literals. The set of all formulas in conjunctive normal form where $|C_i| = N\ i \in 1,...,n$ is called \emph{NCNF}. The intersection of these set with the \emph{SAT} set are called \emph{CNF-SAT} y \emph{NCNF-SAT}. If the context is clear enough the problems will be called \emph{CNF} and \emph{NCNF}\\


	We could define an equal relationship on the set of formulas. Let $F,G$ be formulas. Then $F= G$ if it happens that for each $\alpha$ an assignment such that $F\alpha = 1$ then $G\alpha = 1$ and $G\alpha = 1$ then $F\alpha = 1$

\begin{proposition} 
	The given equal relationship is a equivalence relationship.
\end{proposition}
\begin{proof}
	All three properties follows from the equivalent properties on the constants.

        \end{proof}
	
	
	We could define a partial order relation between the formulas.	Let $F,G$ be formulas. Then $F\le G$ if it for each $\alpha$ an assignment such that $F\alpha = 1$ then $G\alpha = 1$. \\
	
	
\begin{proposition} 
	The given equal relationship is an equivalence relationship.
\end{proposition}
\begin{proof}
As we then could see each class of equivalent as the set of assignment that satisfies all of the clauses, this property arises from the order given by the inclusion on sets.\\
\end{proof}
	
	
\begin{lemma}
	For every \emph{SAT} formula there is an associated circuit.
\end{lemma}
\begin{proof}
	Every operator can be seen as a gate and every variable as an input.\\
\end{proof}
	
\begin{theorem}[Tseitin \cite{tseitin1983complexity}]
  There is a \emph{3-CNF} formula on each equivalent class. Moreover, given an element $F$  there is a equivalent formula $G$  in \emph{3-CNF} which could be done in polynomial time. 
\end{theorem}

\begin{proof}
We will show that for every circuit with $n$ inputs and $m$ binary gates there is a formula in \emph{3-CNF}  that could be constructed in polynomial time in $n$ and $m$. Then, given a formula we will work with it considering it associated circuit.\\

We will construct the formula considering variables $x_1,...,x_n$ that will represents the inputs and $y_1,...,y_n$ that will represents the output of each gate. 

$$ G = (y_1) \wedge \bigwedge_{i=1}^m (y_i \leftrightarrow f_i(z_{i,1},z_{i,2}))$$

Where $f_i$ represents the formula associated to the $i$-gate, $z_{i,1},z_{i,2}$ each of the two inputs of the $i$-gate, whether they are $x_-$ or $y_-$ variables. This formula is not \emph{3-CNF} yet, but for each configuration being $f_i$ a Boolean operator there would be a \emph{3-CNF} equivalent.

\begin{itemize}
	\item $z \leftrightarrow( x \vee y )  = \neg  ( z \vee  x \vee y    ) \vee (z \wedge ( x \vee y )  ) = \neg  ( z \vee  x \vee y    ) \vee (z \wedge x)  \vee (z \wedge y ) =$\\$= ( \neg  z \wedge  \neg  x \wedge \neg   y    ) \vee (z \wedge x)  \vee (z \wedge y )  =$$
	 (\neg  z \vee (z \wedge x)  \vee (z \wedge y ))  \wedge  
	 (\neg  x \vee (z \wedge x)  \vee (z \wedge y )) \wedge
	 (\neg  y \vee (z \wedge x)  \vee (z \wedge y ))   =
	 (\neg  z \vee x  \vee y )  \wedge  
	 (\neg  x \vee z  ) \wedge
	 (\neg  y \vee z ) $   
	\item $z \leftrightarrow( x \wedge y ) = \neg ( z \vee ( x \wedge y )) \vee (z \wedge ( x \wedge y )) = (z\wedge x \wedge y ) \vee  (\neg  z\wedge \neg  x \wedge \neg  y )  =$\\$\ \ \ ((z\vee  (\neg  z\wedge \neg  x \wedge \neg  y )  ) \wedge (x \vee  (\neg  z\wedge \neg  x \wedge \neg  y )  ) \wedge (y\vee  (\neg  z\wedge \neg  x \wedge \neg  y )  ) ) = (\neg  x \vee z) \wedge (\neg  y \vee z ) \wedge (\neg  z \vee x ) \wedge (\neg  y \vee x ) \wedge(\neg  z\vee y )\wedge (\neg  x\vee y )$
	
	\item $z \leftrightarrow( x \leftrightarrow y ) =  \neg ( z \vee ( x \leftrightarrow y ) ) \vee (z \wedge ( x \leftrightarrow y ) = \neg ( z \vee (\neg  x \wedge \neg  y) \vee (x \wedge y)) \vee (z \wedge(\neg  x \wedge \neg  y) \vee (x \wedge y))  )=(\neg  z \wedge \neg  (\neg  x \wedge \neg  y) \wedge \neg  (x \wedge y)) \vee (z \wedge(\neg  x \wedge \neg  y) \vee (x \wedge y))  )=(\neg  z \wedge  (x \vee  y) \wedge (\neg  x \vee \neg  y)) \vee (z \wedge(\neg  x \wedge \neg  y) \vee (x \wedge y))  )=z \vee ( \neg  x \wedge \neg  y) = (\neg x \vee \neg y \vee z) \wedge (\neg x \vee \neg z \vee y) \wedge (y \vee z \vee x) \wedge (y \vee \neg y \vee x) \wedge (\neg z \vee z \vee x) \wedge (\neg z \vee \neg y \vee x)$
	
	\item $z \leftrightarrow( x \oplus y ) =  z \leftrightarrow(\neg  x \leftrightarrow y )  $
	
\end{itemize}
	In the last item we use the third one.
	
\end{proof}
This result is important because, now we could be able to talk only about \emph{3-CNF} formulas. The fact that they are reachable on polynomial time is important because it means it could be done efficiently. Should this be impossible it will not be of much relevance in practice, as we yearn to solve this problem as efficient as possible (in fact, as polynomial as possible). This results implies that if we know how to solve \emph{3-CNF} then we will be able to solve 'full'SAT problems.
\begin{definition}
	An assignment is called autark for a formula $F\in$\emph{CNF} if for every clause $C \in F$ it happens that if $Var(C) \cap Var(\alpha) \ne \emptyset $ then $C\alpha = 1$, in other words it satisfies all clauses that it 'touches'. 
\end{definition}

The use of this definition is self-evident, as it would simplifies the problem of resolving a \emph{CNF} clause. The strategy would be simple as obvious: try to make every clause positive. These assignment will give simplifications of the problem, and enabling a good method for these search will be useful.\\ 

Should it happen that we got an algorithm for autarks clauses, and iterating it, we could find a solution of any given formula. Finding a polynomial algorithm that find whether it exists any non-empty autark formula and provide it, we could be able of proving that NP = P, as we could solve SAT applying this algorithm iteratively. Anyway, trying to find simple autark assignment, i.e. assignment with not many variables,i s a good praxis.

\begin{proposition} We could reduce the SAT-CNF problem to the Autark-Finding problem.
\begin{proof} Suppose that an algorithm such that if it exists any autark it return one of them, and end with an error code otherwise is given.  \\

Given a formula $F$, if there is not an autark then there is no solution for the SAT problem. If it find an Autark-assignment $\alpha$then we apply the same algorithm to $\alpha(F)$. Also, as it happens that $|Var(\alpha(F))|<|Var(F)|$ so we would only apply the algorithm finitely many times. Also, $F$ will be solvable if, and only if, $F\alpha$ is solvable.	\\

Moreover, as checking if an assignment is autark is linear on the number of clauses, then it made the autark-finding problem NP-Complete(NP-C further on).
\end{proof}
	
\end{proposition}



\begin{proposition}
	Given $F \to G$ a tautology, there exists a formula $I$ such that $Var(I) = Var(F)\cap Var(G)$ and both $F\to I$ and $I \to G$ are tautologies. It is not known an polynomial algorithm to solve this problem. 
\end{proposition}

\begin{proof} Let $\{x_1,...,x_k\} = Var(F)\cup Var(G)$ then we will make $I$ by defining its truth table the following way: Given an assignment $\alpha$:
\[   
I\alpha = 
     \begin{cases}
       1&\quad\text{if $\alpha$ could be extended to an assignment that \emph{satisfies} $F$}, \\
       0&\quad\text{if $\alpha$ could be extended to an assignment that \emph{nullifies} $G$},\\
     * &\quad\text{otherwise.} \\ 
     \end{cases}
\]

Where * mean that it could be either 0 or 1.  This is well defined because if for an arbitrary happens that $G\alpha = 0$ then $F\alpha = 0$.

For every $\beta$ an assignment such that $Var(\beta) = Var(F)\cup Var(G)$ then if $\beta(F) = 1$ then $\beta(I) = 1$ so $F \to I$  is a tautology. Similarly it can not happens that $I\beta = 1 $ and $G\beta = 0$, because the second it will imply that   $I\beta = 0$.\\

For the last part we will show that, should it happens that a polynomial algorithm for interpolation will mean that $NP = P$, as it is done in \cite{schoning2013satisfiability}.\\


{\color{red} TODO }
\end{proof}



\section{Satisfiability by Combinatorics}


To get an intuition about the way that unsolvable clauses are, we gonna state some simple result about combinatorics and resolution. This will give the reader an idea of how these formulas should be. \\

Firstly, it is easy to break a big clause on some smaller ones, adding one another on this fashion: Suppose we got two positive integers $n,m$ such that $m < n$ a clause $x_1\vee x_2 \vee ... \vee x_n$ we could split it into two parts $x_1\vee x_2  \vee ... \vee x_{m-1} \vee y, \neg y \vee x_m \vee ... \vee x_n$. Also given the same clause with a given length $n$ we could enlarge it one variable adding $ x_1 \vee ... \vee x_n \vee y$ and $ x_1 \vee ... \vee x_n \vee \neg y$. Note that to enlarge a clause from a length $m$ to a length $n>m$ we would generate $2^{n-m}$ clauses.


\begin{proposition}
	Let $F$ be a CNF formula which has exactly k literals, if $|F| < 2^k$ then $F$ is satisfiable.
\end{proposition}
\begin{proof}
	Let $n = Var(F)$, it happens that $n > k$. For each clause $C \in F$ there are $2^{n-k}$ assignment that falsify $F$, so in total there could be strictly less than $2^k \cdot 2^{n-k} = 2^n.$ Therefore it exists an assignment that assign all variables and not falsifies the formula $F$.
\end{proof}
\begin{proposition}
	Let $F=\{C_1,...,C_n\}$ be a CNF formula. If $\sum_{j=1}^m 2^{-|C_j|}<1,$ then $F$ is satisfiable.
\end{proposition}
\begin{proof}
	Enlarging clauses the way it is explained to the maximum length $k$ ans applying the previous result.
\end{proof}

Following this idea we could define the weight of a clause $C\in F$ as $$\omega(C) = 2^{-|C|} $$
being this the probability that a uniform-random assignment violates this clause. 

\begin{corollary}
	For a formula in CNF, if the sum of the weights of the clauses is less than one then the formula is satisfiable.
\end{corollary}


\begin{proof}
For this task, we will give a probabilistic algorithm, only to prove that it will end with a big probability. Probabilistic (and heuristics) approaches to the problem would prove later on to be really useful. Let $F$ be a CNF formula regard as a clause set.
\end{proof}



\begin{definition}
Let $F$ be a CNF formula. It is said to be minimally unsatisfiable if:
\begin{itemize}
	\item $F$ is unsatisfiable.
	\item $F\backslash \{C\}$ is satisfiable $\forall C \in F$.
\end{itemize}
\end{definition}

\begin{lemma}
Let F be a set of clauses. If for every subset $G$ of $F$ it holds that $|G|\le |Var(G)|$, then $F$ is satisfiable.

\end{lemma}

\begin{proposition}
	Minimally unsatisfiable, then $|F| > Var(F)$.
\end{proposition}

\begin{proof}
	
\end{proof}


%% \section{Cosas que seguir haciendo}
%% \begin{itemize}
%% 	\item NP-Completennes.
%% 	\item Hacer programa para autarks y programa para tranformar en 3-CNF.
%% 	\item Acabar con combinations restrictions.
%% \end{itemize}


%% \section{Sources}
%% \begin{thebibliography}{9}


%% \bibitem{Main} Schöning, U., Torán, J. (2013). The satisfiability problem. Berlin: Lehmanns.
%% \end{thebibliography}

 
 



