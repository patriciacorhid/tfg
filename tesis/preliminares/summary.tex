%*******************************************************
% Summary
%*******************************************************

\newpage



\chapter*{Summary}
\addcontentsline{toc}{chapter}{Summary}
\section*{Brief Summary}



\section*{Resumen Extendido}

Durante la redacción de este trabajo estudiaremos un campo multidisciplinar que involucra por lo general tanto consideraciones propias de los intereses de las matemáticas como de los intereses del campo de la ingeniería informática. Por lo tanto, es difícil distinguir a grandes rasgos que partes son de mayor interés para un lector que pertenezca a solo una de las disciplinas. Sin embargo intentaremos diferenciar, en la medida de lo posible, que partes son de interés para cada una de dichas disciplinas. En caso de que esto no sea posible, justificaremos el interés que cada ciencia tiene.\\

La primera parte introduce el problema y estudia su complejidad. Procedemos a explicar que realizamos en cada capitulo.\\

\textbf{Capítulo 1:} En este capitulo exponemos los fundamentos de la lógica. El objetivo es introducir con la máxima formalidad posible el marco de trabajo durante todo el proyecto. 
Empezamos definimos el Álgebra Booleana como un retículo distributivo. Posteriormente, demostramos el teorema de Knaster-Tarski para retículos completos. Definimos a continuación la Lógica Proposicional como un Lenguaje Formal. De este modo primero especificamos la sintaxis y la semántica de este 'idioma'. Definimos una semántica basada en asignaciones y asignaciones parciales. 
Consideramos que este capítulo suscita un interés principal en el campo de la matemática. Sin embargo es recomendado que todo lector lo lea para poder comprender en toda su extensión los capítulos subsecuentes.\\

\textbf{Capítulo 2:} En este capítulo definimos formalmente lo que es un problema, e introducimos el problema SAT como un problema de decisión sobre el lenguaje de la lógica proposicional. Definimos las variaciones de este problema que suscitan más interés, y que estudiaremos en el siguiente problema. Para acabar el capítulo demostramos como se realiza un certificado de ejecución de SAT. Para ello demostramos la completitud del la regla de resolución. Por último hablamos sobre los problemas de satisfacción de restricciones. Consideramos que la definición del problema es de interés para ambas disciplinas. La parte de completitud es más interesante para un punto de vista matemático, así como la última sección es más interesante para la ingeniería informática.\\

\textbf{Capítulo 3:}  En este capítulo consideramos que este capítulo debe ser de interés para ambas disciplinas. En las primeras tres secciones definimos los modelos de computación, las clases de complejidad y la teoría de completitud algorítmica. En la tercera sección llega a su punto algido introducimos el Teorema de Cook-Levin.
En la última sección realizamos un desarrollo de teorías asociadas con el problema SAT. En particular estudiamos la acción de grupos de permutaciones y negaciones sobre fórmulas CNF, desarrollamos el teorema de Tseitin, consideramos CO-NP completitud y, para terminar, consideramos FSAT y la relación entre los métodos constructivos y no constructivos de resolución.\\


Proseguimos explicando la siguiente parte. En esta, nos centramos en los algoritmos de resolución. Dividimos nuestro estudio de los algoritmos en tres capítulos.\\


\textbf{Capítulo 4:} En este capítulo estudiamos los algoritmos de resolución sobre conjuntos restringidos de SAT donde se pueden conseguir mejores resoluciones. Empezamos considerando restricciones aplicando nociones de combinatoria. Esta sección es naturalmente para el lector informático. Entre los métodos destacamos para el lector matemático la aplicación que explicamos del Lema Local de Lovasz, asociando un grafo de adyacencia a cada fórmula para poder aplicar el resultado. Proseguimos explicando los famosos casos de 2SAT y HORNSAT. \\

\textbf{Capítulo 5:} En este capítulo estudiamos los algoritmos completos de resolución de SAT. Las dos estrategias principales estudiadas son el algoritmo DPLL y sus mejoras, incluyendo el algoritmo de Monien-Speckenmeyer y la explicación de los muy extendidos algoritmos CDCL. Tomando otro enfoque explicamos el método de búsqueda local ajustado a SAT, así como lo mejoramos aplicando la teoría de código de cubrimientos. Esta sección consideramos que tiene un interés natural para el ámbito de la ingeniería informática. Desde el ámbito de las matemáticas el interés reside en el análisis de la complejidad de los algoritmos.\\

\textbf{Capítulo 6:} En este capítulo 



\endinput
